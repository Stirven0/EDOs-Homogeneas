% introduccion.tex
\section{Introducción}

Las ecuaciones diferenciales homogéneas con coeficientes lineales representan una clase fundamental de ecuaciones que modelan fenómenos dinámicos en múltiples disciplinas de la ingeniería. En el contexto de la Ingeniería de Sistemas, estas ecuaciones son particularmente relevantes para describir y analizar el comportamiento temporal de sistemas complejos, desde redes de computadoras hasta procesos de manufactura automatizada.

La importancia de estas ecuaciones radica en su capacidad para capturar relaciones causa-efecto en sistemas donde las tasas de cambio dependen linealmente del estado actual. Por ejemplo, en sistemas de control automático, las ecuaciones diferenciales homogéneas permiten modelar la respuesta temporal de controladores PID, la dinámica de sistemas mecánicos amortiguados o la propagación de señales en redes de comunicación.

En el ámbito de la simulación de procesos, estas ecuaciones permiten predecir el comportamiento de sistemas bajo diferentes condiciones iniciales, facilitando la optimización de parámetros y la identificación de posibles fallos antes de la implementación física. La capacidad de resolver analíticamente estas ecuaciones proporciona una ventaja significativa en términos de comprensión del sistema y predicción de su comportamiento a largo plazo.

Este documento presenta una revisión sistemática de los conceptos teóricos fundamentales relacionados con ecuaciones diferenciales homogéneas con coeficientes lineales, sus métodos de solución y aplicaciones prácticas específicas en Ingeniería de Sistemas. El enfoque se centra en proporcionar herramientas matemáticas que permitan a los ingenieros modelar, analizar y predecir el comportamiento de sistemas dinámicos complejos.