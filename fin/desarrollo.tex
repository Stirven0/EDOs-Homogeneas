% desarrollo.tex
\section{Desarrollo}

\subsection{Conceptos Fundamentales}

Una ecuación diferencial homogénea con coeficientes lineales es una ecuación de la forma
\[
a_n\frac{d^n y}{dx^n}+a_{n-1}\frac{d^{n-1}y}{dx^{n-1}}+\cdots+a_1\frac{dy}{dx}+a_0y=0,
\]
donde $a_0,\dots,a_n$ son constantes.

\subsection{Ecuaciones de Segundo Orden}

Para el caso particular de segundo orden:
\[
a\frac{d^2y}{dx^2}+b\frac{dy}{dx}+cy=0.
\]
La ecuación característica $ar^2+br+c=0$ determina la solución:

\begin{itemize}
\item \textbf{Raíces reales distintas} ($b^2-4ac>0$):
\[
y(x)=C_1e^{r_1x}+C_2e^{r_2x}.
\]

\item \textbf{Raíces reales repetidas} ($b^2-4ac=0$):
\[
y(x)=(C_1+C_2x)e^{rx}.
\]

\item \textbf{Raíces complejas conjugadas} ($b^2-4ac<0$):
\[
y(x)=e^{\alpha x}\bigl(C_1\cos(\beta x)+C_2\sin(\beta x)\bigr).
\]
\end{itemize}

\subsection{Aplicación en Sistemas de Control}

Un sistema masa-resorte-amortiguador se modela mediante
\[
m\frac{d^2x}{dt^2}+c\frac{dx}{dt}+kx=0,
\]
donde $m$ es la masa, $c$ el coeficiente de amortiguamiento y $k$ la constante del resorte. La solución describe la respuesta natural y permite analizar estabilidad y tiempo de asentamiento en controladores.

\subsection{Análisis de Redes de Comunicación}

En un modelo M/M/1 la dinámica de la probabilidad $P_n(t)$ de tener $n$ clientes satisface
\[
\frac{dP_n(t)}{dt}=\lambda P_{n-1}(t)-(\lambda+\mu)P_n(t)+\mu P_{n+1}(t),
\]
que en régimen estacionario conduce a un sistema homogéneo útil para estudiar la estabilidad de la red.