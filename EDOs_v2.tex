% ------------------------------------------------------------------------------
%  ECUACIONES DIFERENCIALES HOMOGÉNEAS Y DE COEFICIENTES CONSTANTES
%  Trabajo estudiantil – Normas APA 7ª ed.
% ------------------------------------------------------------------------------
\documentclass[stu,12pt,donotrepeattitle]{apa7}
% ------------------------------------------------------------------------------
% PAQUETES ESENCIALES
% ------------------------------------------------------------------------------
\usepackage[spanish,es-noindentfirst,es-noshorthands]{babel} % Español global
\usepackage[utf8]{inputenc}                                  % Codificación UTF-8
\usepackage{amsmath,amssymb,amsfonts}                        % Matemáticas
\usepackage{graphicx}                                        % Figuras (si se necesitan)
\usepackage{booktabs}                                        % Tablas profesionales
\usepackage{enumitem}                                        % Listas personalizadas
\usepackage{lipsum}                                          % Texto de relleno (opcional)
% ------------------------------------------------------------------------------
% METADATOS APA 7 (ESTUDIANTIL)
% ------------------------------------------------------------------------------
\title{Ecuaciones Diferenciales Homogéneas y de Coeficientes Constantes}
\shorttitle{ED homogéneas y coef. constantes}
\author{María José Pérez López}
\affiliation{Universidad Nacional de Ingeniería, Facultad de Ciencias Matemáticas}
\course{Matemática IV: Ecuaciones Diferenciales}
\professor{Dr. Carlos A. Rodríguez}
\duedate{15 de octubre de 2025}
\abstract{
  El presente documento aborda el estudio de dos familias fundamentales de ecuaciones diferenciales ordinarias: las ecuaciones homogéneas y las lineales de coeficientes constantes. Se exponen los conceptos teóricos básicos, los métodos de solución más utilizados y se ilustran con ejemplos completos. En el primer caso, se describe la técnica de sustitución \(y=ux\) que reduce la ecuación a una de variables separables. En el segundo, se detalla la construcción del polinomio característico y la clasificación de la solución general según la naturaleza de sus raíces (reales distintas, repetidas o complejas). Los procedimientos se acompañan de desarrollos paso a paso que permiten al lector replicar los resultados. El trabajo concluye destacando la utilidad de ambos métodos en modelación matemática e ingeniería.}

\begin{document}
\maketitle

% ------------------------------------------------------------------------------
% INTRODUCCIÓN
% ------------------------------------------------------------------------------
\section{Introducción}
Las ecuaciones diferenciales constituyen una herramienta esencial para describir fenómenos que evolucionan en el tiempo o en el espacio. En particular, las \emph{ecuaciones diferenciales homogéneas} y las \emph{ecuaciones lineales con coeficientes constantes} aparecen en vibraciones mecánicas, circuitos eléctricos, dinámica de poblaciones y transferencia de calor, entre otros campos.

Una ecuación diferencial de primer orden se dice \emph{homogénea} si puede escribirse en la forma \(y'=f(y/x)\), mientras que una ecuación lineal de orden \(n\) con coeficientes constantes tiene la expresión
\[
a_n y^{(n)}+a_{n-1}y^{(n-1)}+\dots+a_1y'+a_0y=g(t),
\]
con \(a_i\in\mathbb{R}\). Cuando \(g(t)\equiv 0\), la ecuación es \emph{homogénea} en el sentido lineal.

El objetivo de este trabajo es presentar los métodos clásicos de resolución para ambos tipos de ecuaciones, ilustrando los procedimientos con ejemplos completos que guíen al estudiante en la aplicación práctica de las técnicas.

% ------------------------------------------------------------------------------
% DESARROLLO
% ------------------------------------------------------------------------------
\section{Desarrollo}

\subsection{Ecuaciones diferenciales homogéneas}
Una ecuación de primer orden
\begin{equation}\label{eq:hom1}
M(x,y)\,dx+N(x,y)\,dy=0
\end{equation}
se denomina \emph{homogénea} si \(M\) y \(N\) son funciones homogéneas del mismo grado \(k\), es decir,
\[
M(tx,ty)=t^kM(x,y),\qquad N(tx,ty)=t^kN(x,y),\quad\forall t\neq 0.
\]
Bajo esta condición, la sustitución
\begin{equation}\label{eq:sust}
y=ux\quad(\text{o bien }x=vy)
\end{equation}
transforma \eqref{eq:hom1} en una ecuación de variables separables \cite{Zill2017}.

\subsubsection*{Ejemplo resuelto}
Resuelva la ecuación
\[
(x^2+y^2)\,dx-2xy\,dy=0.
\]

\textbf{Paso 1:} Verificar homogeneidad.\\
Sean \(M=x^2+y^2\) y \(N=-2xy\). Para \(t\neq 0\):
\[
M(tx,ty)=t^2x^2+t^2y^2=t^2M(x,y),\qquad
N(tx,ty)=-2(tx)(ty)=t^2(-2xy)=t^2N(x,y).
\]
Ambas son homogéneas de grado 2.

\textbf{Paso 2:} Aplicar la sustitución \(y=ux\), de donde \(dy=u\,dx+x\,du\).\\
Sustituyendo:
\[
(x^2+u^2x^2)\,dx-2x(ux)(u\,dx+x\,du)=0
\;\Longrightarrow\;
x^2(1+u^2)\,dx-2u x^2(u\,dx+x\,du)=0.
\]

\textbf{Paso 3:} Simplificar y separar variables.
\[
x^2\bigl[(1+u^2)-2u^2\bigr]\,dx-2u x^3\,du=0
\;\Longrightarrow\;
x^2(1-u^2)\,dx=2u x^3\,du.
\]
Dividiendo entre \(x^2(1-u^2)x\) (asumiendo \(x\neq 0\) y \(u^2\neq 1\)):
\[
\frac{dx}{x}=\frac{2u}{1-u^2}\,du.
\]

\textbf{Paso 4:} Integrar.
\[
\int\frac{dx}{x}=\int\frac{2u}{1-u^2}\,du
\;\Longrightarrow\;
\ln|x|=-\ln|1-u^2|+C.
\]

\textbf{Paso 5:} Deshacer el cambio \(u=y/x\).
\[
\ln|x|=-\ln\left|1-\frac{y^2}{x^2}\right|+C
\;\Longrightarrow\;
\ln|x|+\ln\left|\frac{x^2-y^2}{x^2}\right|=C
\;\Longrightarrow\;
\ln\left|\frac{x^2-y^2}{x}\right|=C.
\]
Exponentiando:
\[
\frac{x^2-y^2}{x}=K,\qquad K=\pm e^C.
\]
Finalmente,
\[
x^2-y^2=Kx\quad\Longrightarrow\quad y^2=x^2-Kx.
\]

\subsection{Ecuaciones diferenciales lineales con coeficientes constantes}
Una ecuación lineal de orden \(n\)
\begin{equation}\label{eq:lincc}
a_n y^{(n)}+a_{n-1}y^{(n-1)}+\dots+a_1y'+a_0y=0,\qquad a_i\in\mathbb{R},\;a_n\neq 0,
\end{equation}
se resuelve mediante el \emph{polinomio característico}
\[
P(r)=a_nr^n+a_{n-1}r^{n-1}+\dots+a_1r+a_0=0.
\]
La solución general depende de la naturaleza de las raíces de \(P(r)\) \cite{Boyce2012}:

\begin{itemize}[leftmargin=*,nosep]
\item \textbf{Raíces reales distintas} \(r_1,\dots,r_n\):
\[
y(t)=C_1e^{r_1t}+\dots+C_ne^{r_nt}.
\]

\item \textbf{Raíces reales repetidas} \(r\) con multiplicidad \(m\):
\[
e^{rt},\;te^{rt},\;\dots,\;t^{m-1}e^{rt}.
\]

\item \textbf{Raíces complejas simples} \(\alpha\pm i\beta\):
\[
e^{\alpha t}\cos(\beta t),\qquad e^{\alpha t}\sin(\beta t).
\]
\end{itemize}

\subsubsection*{Ejemplo resuelto}
Resuelva
\[
y''-4y'+13y=0.
\]

\textbf{Paso 1:} Escribir el polinomio característico.
\[
r^2-4r+13=0.
\]

\textbf{Paso 2:} Calcular las raíces.
\[
r=\frac{4\pm\sqrt{(-4)^2-4\cdot 13}}{2}
=\frac{4\pm\sqrt{16-52}}{2}
=\frac{4\pm\sqrt{-36}}{2}
=2\pm 3i.
\]

\textbf{Paso 3:} Construir la solución general.\\
Con \(\alpha=2\) y \(\beta=3\):
\[
y(t)=e^{2t}\bigl(C_1\cos 3t+C_2\sin 3t\bigr).
\]

% ------------------------------------------------------------------------------
% CONCLUSIÓN
% ------------------------------------------------------------------------------
\section{Conclusión}
Se han revisado dos clases de ecuaciones diferenciales ampliamente utilizadas en matemática aplicada: las homogéneas de primer orden y las lineales de coeficientes constantes. En el primer caso, la sustitución \(y=ux\) reduce el problema a una ecuación de variables separables, mientras que en el segundo el polinomio característico proporciona la estructura de la solución general a partir de sus raíces. Ambos métodos destacan por su eficiencia y versatilidad, permitiendo modelar sistemas físicos y procesos de ingeniería con precisión. El dominio de estas técnicas constituye una base indispensable para abordar problemas más complejos, como sistemas acoplados o ecuaciones no homogéneas.

% ------------------------------------------------------------------------------
% REFERENCIAS
% ------------------------------------------------------------------------------
\begin{thebibliography}{}

\bibitem{Boyce2012}
Boyce, W. E., \& DiPrima, R. C. (2012). \emph{Elementary Differential Equations and Boundary Value Problems} (10.ª ed.). Wiley.

\bibitem{Zill2017}
Zill, D. G. (2017). \emph{Ecuaciones diferenciales con aplicaciones de modelado} (11.ª ed.). Cengage Learning.

\end{thebibliography}

\end{document}
% ------------------------------------------------------------------------------