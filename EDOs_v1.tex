% !TEX program = pdflatex
\documentclass[stu]{apa7} % opción 'stu' para trabajo estudiantil
\usepackage[spanish]{babel}
\usepackage[utf8]{inputenc}
\usepackage[T1]{fontenc}
\usepackage{amsmath,amsfonts,amssymb}
\usepackage{microtype}
\usepackage{geometry}
\usepackage{filecontents}

% --- Bibliografía (archivo .bib embebido) ---
\begin{filecontents}{refs.bib}
@book{boyce2017,
  author    = {William E. Boyce and Richard C. DiPrima},
  title     = {Elementary Differential Equations and Boundary Value Problems},
  year      = {2017},
  publisher = {Wiley},
  address   = {Hoboken, NJ}
}
@book{kreyszig2011,
  author    = {Erwin Kreyszig},
  title     = {Advanced Engineering Mathematics},
  year      = {2011},
  edition   = {10th},
  publisher = {Wiley},
  address   = {Hoboken, NJ}
}
@misc{apa7cls,
  author = {{Daniel A. Weiss}},
  title  = {The \texttt{apa7} \LaTeX{} class --- documentation and examples},
  year   = {2022},
  howpublished = {\url{https://ctan.org/pkg/apa7}}
}
@misc{apa_student_guide,
  author = {{American Psychological Association}},
  title = {7th Edition Student Paper Setup Guide},
  year = {2020},
  url = {https://apastyle.apa.org/instructional-aids/student-paper-setup-guide.pdf}
}
\end{filecontents}

\usepackage[style=apa,backend=biber]{biblatex}
\addbibresource{refs.bib}

% Forzar etiquetas en español: título del resumen y referencias
\addto\captionsspanish{%
  \renewcommand{\abstractname}{Resumen}%
  \renewcommand{\refname}{Referencias}%
}
% Si se requiere, ajustar strings de biblatex (opcional)
\DefineBibliographyStrings{spanish}{%
  references = {Referencias},
}

% Metadatos APA (puedes modificar estos campos si lo deseas)
\title{Ecuaciones Diferenciales Homogéneas y de Coeficientes Constantes}
\shorttitle{Ecuaciones diferenciales: homog. y coef. const.}
\author{Harold S. González I.}
\affiliation{Facultad de Ingeniería, Universidad Ejemplo}
\course{Cálculo de Ecuaciones Diferenciales (2051)}
\professor{Dr. María Pérez}
\duedate{28 de septiembre de 2025}

\begin{document}
\maketitle

\abstract{
Este trabajo presenta una exposición clara y autónoma sobre dos familias
fundamentales de ecuaciones diferenciales ordinarias: las ecuaciones
diferenciales homogéneas (primero orden, con métodos de sustitución)
y las ecuaciones diferenciales lineales de coeficientes constantes
(segundo orden como caso representativo). Se definen los conceptos
clave, se justifica la importancia de cada tipo en aplicaciones
matemáticas e ingenieriles, y se desarrollan métodos de solución con
ejemplos resueltos paso a paso para ilustrar la técnica. En la sección
de ecuaciones homogéneas se explica y aplica la sustitución
\(y=ux\) para reducir la ecuación a variables separables; en la sección
de coeficientes constantes se construye la ecuación característica y se
muestra cómo las raíces reales, repetidas o complejas determinan la
forma de la solución general. Finalmente, se discuten brevemente
implicaciones y aplicaciones prácticas de ambos métodos. \par
}

\section{Introducción}
Las ecuaciones diferenciales ordinarias (EDO) modelan relaciones entre
funciones y sus derivadas; son herramientas esenciales en matemáticas
aplicadas e ingeniería para describir fenómenos dinámicos. En este
documento nos centramos en dos clases ampliamente estudiadas:
\emph{ecuaciones diferenciales homogéneas} (principalmente de primer
orden, resolubles por sustituciones tipo \(y=ux\)) y
\emph{ecuaciones lineales con coeficientes constantes} (a menudo de
orden superior, tratadas mediante la ecuación o polinomio
característico). El objetivo es ofrecer definiciones, métodos y ejemplos
resueltos paso a paso que permitan comprender y aplicar estas técnicas
en problemas reales. \par

% Citar guía sobre clase apa y cómo estructurar student paper
El formato y la portada de este documento siguen las recomendaciones de
la clase \texttt{apa7} para trabajos estudiantiles y las guías
oficiales de la APA para trabajos de estudiantes. \par

\section{Ecuaciones diferenciales homogéneas}
\subsection{Definición y teoría}
Una ecuación diferencial de primer orden
\[
\frac{dy}{dx} = f(x,y)
\]
se dice \emph{homogénea} si la función \(f\) puede expresarse como
\(f(x,y)=G\!\left(\frac{y}{x}\right)\) (o equivalente) —es decir, \(f(tx,ty)=f(x,y)\)
para todo \(t\neq 0\). En la práctica esto significa que las variables
se pueden agrupar por la razón \(y/x\) y se emplea la sustitución
\(y = u x\) (o \(x = v y\)) para reducir la EDO a una de variables
separables \parencite{kreyszig2011}. Esta técnica y otras sustituciones
están descritas en los manuales y notas de ODE contemporáneas. \parencite{apa_student_guide, apa7cls}

\subsection{Método de la sustitución \(y=ux\)}
Sea la sustitución \(y = u(x)\,x\). Entonces
\[
\frac{dy}{dx} = u + x\frac{du}{dx}.
\]
Sustituyendo en la EDO original de la forma \(\frac{dy}{dx}=G\!\left(\frac{y}{x}\right)\)
se obtiene una ecuación en \(u\) y \(x\) que suele ser separable o
integrable. Véase la sección dedicada a sustituciones para más detalles. \parencite{kreyszig2011, apa_student_guide}

\subsection{Ejemplo resuelto (paso a paso)}
Resolvamos la ecuación
\begin{equation}\label{eq:ej-homog}
\frac{dy}{dx} = \frac{x+y}{x}.
\end{equation}

\textbf{Paso 1 (verificar homogeneidad).} Observe que
\(\dfrac{x+y}{x}=1+\dfrac{y}{x}\), que depende sólo de \(y/x\); por lo
tanto la ecuación es homogénea de grado cero.

\textbf{Paso 2 (sustitución).} Tomamos \(y = u x\), con \(u = u(x)\).
Entonces
\[
\frac{dy}{dx} = u + x\frac{du}{dx}.
\]

\textbf{Paso 3 (sustituir en la EDO).} Reemplazando en
\eqref{eq:ej-homog}:
\[
u + x\frac{du}{dx} = 1 + \frac{y}{x} = 1 + u.
\]

\textbf{Paso 4 (simplificar).} Cancelamos \(u\) en ambos lados:
\[
x\frac{du}{dx} = 1 \quad\Longrightarrow\quad \frac{du}{dx} = \frac{1}{x}.
\]

\textbf{Paso 5 (integrar).} Integrando respecto a \(x\):
\[
u(x) = \int \frac{1}{x}\,dx = \ln|x| + C,
\]
donde \(C\) es la constante de integración.

\textbf{Paso 6 (volver a la variable original).} Dado que
\(y = u x\),
\[
y(x) = x\bigl(\ln|x| + C\bigr) = x\ln|x| + Cx.
\]

\textbf{Comentario:} Esta solución es la familia general de soluciones
derivada por la sustitución \(y=ux\). El procedimiento mostrado es
típico para EDOs homogéneas de primer orden. \parencite{kreyszig2011, boyce2017}

\section{Ecuaciones diferenciales lineales con coeficientes constantes}
\subsection{Definición y teoría}
Una ecuación diferencial lineal de orden \(n\) con coeficientes
constantes tiene la forma
\[
a_n \frac{d^n y}{dx^n} + a_{n-1}\frac{d^{n-1}y}{dx^{n-1}} + \cdots + a_1 \frac{dy}{dx} + a_0 y = 0,
\]
donde los \(a_i\) son constantes. La técnica estándar para hallar la
solución general consiste en proponer soluciones de la forma
\(y=e^{rt}\), obtener el polinomio característico
\[
a_n r^n + a_{n-1} r^{n-1} + \cdots + a_1 r + a_0 = 0,
\]
y analizar sus raíces (reales distintas, reales repetidas o complejas).
Cada tipo de raíz genera una contribución particular a la solución
general: exponentes reales dan \(e^{rt}\), raíces repetidas generan
factores polinomiales multiplicativos (como \(t e^{rt}\)), y raíces
complejas conjugadas \(\alpha\pm i\beta\) producen soluciones
combinando \(e^{\alpha t}\cos(\beta t)\) y \(e^{\alpha t}\sin(\beta t)\).
Este método está ampliamente explicado en textos clásicos de EDO. \parencite{boyce2017, kreyszig2011, apa7cls}

\subsection{Ejemplo resuelto (polinomio característico)}
Consideremos la EDO de segundo orden
\begin{equation}\label{eq:const-coef}
y'' - 3y' + 2y = 0.
\end{equation}

\textbf{Paso 1 (suposición exponencial).} Proponemos
\(y=e^{rt}\). Entonces \(y'=re^{rt}\), \(y''=r^2 e^{rt}\). Sustituyendo:
\[
r^2 e^{rt} - 3r e^{rt} + 2 e^{rt} = 0 \quad\Longrightarrow\quad (r^2 -3r +2)e^{rt}=0.
\]

\textbf{Paso 2 (polinomio característico).} Descartando la solución
trivial \(e^{rt}\neq 0\):
\[
r^2 - 3r + 2 = 0.
\]

\textbf{Paso 3 (resolver el polinomio).} Factorizamos:
\[
r^2 -3r +2 = (r-1)(r-2)=0 \quad\Rightarrow\quad r_1=1,\; r_2=2.
\]

\textbf{Paso 4 (solución general).} Como las raíces son reales y distintas,
la solución general es
\[
y(t) = C_1 e^{t} + C_2 e^{2t},
\]
donde \(C_1,C_2\) son constantes de integración.

\textbf{Comentarios:} Si las raíces hubieran sido iguales (repetidas),
por ejemplo \(r_1=r_2=r\), la solución general incluiría un factor
polinomial \(t\): \(y=(C_1 + C_2 t) e^{rt}\). Si las raíces hubieran
sido complejas \(\alpha\pm i\beta\), la solución se escribiría con
funciones seno y coseno moduladas por la exponencial \(e^{\alpha t}\).
Véanse desarrollos detallados en textos didácticos sobre ODE. \parencite{boyce2017, kreyszig2011}

\section{Conclusión}
En este documento se presentaron y resolvieron, con ejemplos, dos
familias importantes de ecuaciones diferenciales: las homogéneas de
primer orden (resolubles mediante sustituciones como \(y=ux\)) y las
lineales con coeficientes constantes (resueltas por medio del
polinomio característico). Ambos métodos son fundamentales en el
análisis teórico y en aplicaciones prácticas de ingeniería —por ejemplo
en circuitos eléctricos, sistemas mecánicos y modelos de crecimiento— y
constituyen la base para técnicas más avanzadas (EDOs no lineales,
sistemas de ecuaciones, ecuaciones en derivadas parciales, etc.). El
procedimiento paso a paso mostrado ayuda a consolidar la comprensión
y ofrece una plantilla reproducible para problemas similares. \par

\printbibliography[title={Referencias}]

% También es posible añadir apéndices, tablas comparativas o figuras si se requiere.
\end{document}
