% desarrollo.tex
\newpage
\section{Desarrollo}

\subsection{Conceptos Fundamentales}

Una ecuación diferencial homogénea con coeficientes lineales es una ecuación de la forma

\begin{equation}
    a_n \frac{d^n y}{dx^n} + a_{n-1} \frac{d^{n-1} y}{dx^{n-1}} + \cdots + a_1 \frac{dy}{dx} + a_0 y = 0
\end{equation}

donde $a_0, a_1, \ldots, a_n$ son coeficientes constantes y $y$ es una función de $x$.


\subsection{Ecuaciones de Segundo Orden}

Para el caso particular de ecuaciones de segundo orden, la forma general es:

\begin{equation}
    a \frac{d^2 y}{dx^2} + b \frac{dy}{dx} + c y = 0
\end{equation}

La solución se obtiene mediante la ecuación característica:

\begin{equation}
    ar^2 + br + c = 0
\end{equation}

Las raíces de esta ecuación determinan la forma de la solución general:

\begin{itemize}

\item \textbf{Raíces reales distintas} ($b^2 - 4ac > 0$)
\begin{equation}
    y(x) = C_1 e^{r_1 x} + C_2 e^{r_2 x}
\end{equation}

\item \textbf{Raíces reales repetidas} ($b^2 - 4ac = 0$)
\begin{equation}
    y(x) = (C_1 + C_2 x) e^{rx}
\end{equation}

\item \textbf{Raíces complejas conjugadas} ($b^2 - 4ac < 0$)
\begin{equation}
    y(x) = e^{\alpha x} (C_1 \cos(\beta x) + C_2 \sin(\beta x))
\end{equation}

\end{itemize}

\subsection{Aplicación en Sistemas de Control}

En sistemas de control automático, las ecuaciones diferenciales homogéneas modelan la respuesta natural de sistemas de segundo orden. Consideremos un sistema masa-resorte-amortiguador:

\begin{equation}
    m \frac{d^2 x}{dt^2} + c \frac{dx}{dt} + kx = 0
\end{equation}

donde $m$ es la masa, $c$ es el coeficiente de amortiguamiento, y $k$ es la constante del resorte.

La solución de esta ecuación describe el comportamiento transitorio del sistema, crucial para determinar estabilidad y tiempo de respuesta en sistemas de control.

\subsection{Análisis de Redes de Comunicación}

En el modelado de redes de comunicación, las ecuaciones diferenciales homogéneas describen la dinámica de colas y la propagación de señales. Para un sistema de cola M/M/1, la ecuación de estado puede formularse como:

\begin{equation}
    \frac{dP_n(t)}{dt} = \lambda P_{n-1}(t) - (\lambda + \mu) P_n(t) + \mu P_{n+1}(t)
\end{equation}

En estado estacionario, esta ecuación se simplifica a un sistema homogéneo que permite analizar la estabilidad de la red.
